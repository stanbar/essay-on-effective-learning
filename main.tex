\documentclass{article}

% Language setting
% Replace `english' with e.g. `spanish' to change the document language
\usepackage[english]{babel}

% Set page size and margins
% Replace `letterpaper' with`a4paper' for UK/EU standard size
\usepackage[letterpaper,top=2cm,bottom=2cm,left=3cm,right=3cm,marginparwidth=1.75cm]{geometry}

% Useful packages
\usepackage{amsmath}
\usepackage{graphicx}
\usepackage[colorlinks=true, allcolors=blue]{hyperref}

\title{Learning learning}
\author{Stanisław Barański}
\begin{document}
\maketitle

\section{Why should we learn how to learn}
Meritocracy is a system we live in today; it grants economic goods and political power to people who brings valuable skills and knowledge to society. Therefore, we can look at knowledge acquisition as an investment in our economic well-being. By increasing our learning skills, we achieve better return-of-investment in both time and money spent on learning. 

\section{Mindset}
Learning can not happen without the will to learn. Without believing that we can acquire knowledge, we will sabotage the learning process. Hence, the importance of the correct mindset, namely, the growth mindset, which is a belief that we can learn anything, in contrast to a fixed mindset that says that we cannot learn something because of the lack of some trait. It is just a matter of time, energy and correct techniques used to acquire it. 

Let us consider knowledge to be a baseball, a teacher to be a pitcher, a student to be a catcher and a learning process as throwing the from the pitcher to the catcher~\cite{adler2014read}. A bad pitcher may throw balls that are very hard to catch by the catcher. Therefore, bad teachers can make the process of learning knowledge harder than it may be with different resources. However, if the student is skilful, he will catch a knowledge thrown even by a bad teacher. Hence, the vital role of the correct mindset in learning. Once we realize that we are as responsible for the learning process as the teacher is, we become more emphatic, engaged and forgiving. Knowing that we are playing in the same team as our teacher, not against him, involuntarily forces us to help him by interacting with him. We become more vulnerable to the passion the teacher is trying to share with us. This mindset becomes a win-win situation because students are rewarded with better comprehension and higher grades, while the teacher is rewarded with pleasing experience and possibly new knowledge.

The correct mindset is the fundament; next, we must learn to build on top of it.

\section{Memory}
Memory is the oldest mental skill from which all others derive. Without memory, we would not remember new skills that we have just learned. A person who can not remember is cut off from the knowledge of prior experiences. Without memory, logic could not exist; therefore, we, as a society, could not develop~\cite{csikszentmihalyi1990flow}.

From the physiological point of view, our memory is made of synapses. We create new synapses by learning new things or strengthening existing ones by reusing them.

People have two types of memory, short-term memory (working memory) and long-term memory. Working memory is responsible for gathering and encoding information. Long-term memory is responsible for storing information for later retrieval. Metaphorically speaking, working memory is a blackboard, and long-term memory is a hard drive. Only the information that has been persisted in long-term memory can be called learned. Therefore, in our learning, we should aim for enhancing the process of moving as much information from working memory to long-term memory. This process of moving data from working memory to long-term memory is called consolidation. Consolidation happens when we take breaks, so sleeping, jogging, and exercising is highly advisable, not only to rest, but also to persist the accrued knowledge in long-term memory.


/subsection{Emotions}
Hippocampus is a part of our brain that control access to our memory. This "gatekeeper" decides if the new information will be stored in our memory or not. Hippocampus is influenced by the amygdala, which in turn is influenced by emotions — phenomena called memory enhancement effect. The more emotionally charged event, the better we remember it. If we have no emotions during learning, the amygdala is not stimulated; therefore, our hippocampus is not working correctly; therefore, we are not learning efficiently~\cite{phelps2004human, tyng2017influences}. Facts gathered without any emotional arousal are more vulnerable to disruption~\cite{doi:10.1111/1467-9280.00090}.

On the other hand, "Emotion also facilitates encoding and helps retrieval of information efficiently"~\cite{tyng2017influences}. Therefore, we can boost our learning capabilities by putting ourselves in an emotional state. For example, confusion is a state in which people have increased learning performance because the bad feeling of disequilibrium of disagreeing data arouse the feeling of frustration but also makes him seek equilibrium, which results in enhanced learning~\cite{d2014confusion}. Stress is another common emotional state associated with a learning activity. Depending on intensity and duration, stress can facilitate or impair learning performance. Studies show that a low level of stress can enhance cognitive capabilities, while intense stress has a negative effect, especially on memory retrieval, which is often observed by students underachieving at exams~\cite{vogel2016learning}. 

Apathy, boredom, relaxation, and anxiety are states we should avoid during learning. What we should seek is the optimal state called flow~\cite{csikszentmihalyi1990flow}. Flow is the state where we face high-level challenges complementing our high-level skills. Once we stay in this state for long enough, we call this activity our passion.  Depending on our current state, flow can be achieved from:
\begin{itemize}
    \item apathy or boredom by increasing both challenge level and our skills level;
    \item relaxation by increasing the challenge level;
    \item anxiety by increasing skills level and reducing challenge level.
\end{itemize}

Recognizing and naming our emotions is a hard task itself. Learning how to do it takes time and energy; it requires practising mindfulness, meditation, or yoga. However, the benefits it brings to our life—not only in learning—are priceless. By recognizing our emotions, we are better in understanding ourselves, hence better strategies and decisions we make. Moreover, we can help others understand us.

\section{Metaphore}
The concept of representing phenomena in one domain as an instance of different phenomena in a different domain is called metaphor. 
According to ~\cite{radden2007cognitive},  "Metaphor [...] provides a means of understanding abstract domains such as emotions by relating them to better-known domains and experiences in the physical world."
As we have stated in the first section, the abstract concept of learning has been synthesized with the activity of throwing a baseball. By representing learning in such a metaphor, we can apply our already existing experiences with baseball games to the new concept of learning. Therefore, saving a lot of mental resources on arranging the context of the new knowledge. Suppose we have already seen the baseball game (and the chances are that we have). In that case, we can reuse the prior knowledge that the catcher is as crucial as the pitcher in the baseball game by applying the same understanding to the learning activity—saving time and energy for building the rationalization from scratch.

Metaphor is a robust rhetoric device that allows knowledge extrapolation — acquiring new knowledge connecting what is new to what is known, using an analogy. While learning, we can use metaphor to help our brain connect new knowledge to the existing one. 

Creating metaphors is about noticing patterns between phenomenon in different domains. Metaphors are ubiquitous, but they are hidden. To find them, first, we have to define the properties of the new concept; second, recall another concept that shares the same properties; finally, apply the first concept to the context of the other. For example, when learning computer architecture, we can understand the role of a processor by describing it as the metaphor "the brain of a computer". 

Creating a good metaphor require creativity and energy, but the payoff is worth it. Good metaphor stays in our head for much longer than pure facts, therefore saving our energy in the long term. 

\section{Conclusions}

Learning is a skill that can enhance all areas of our life. If we plan to learn for an extended period of our life—and we probably will—then learning how to learn is an excellent investment in our future.

Learning can be enhanced in multiple ways. Fundamental is the correct mindset, sometimes called the growth mindset, which says that we can, if we want, learn anything. We need to believe that learning anything is just a matter of our effort. 

With the correct mindset, we can focus on optimizing the effort-to-results ratio. The part of the brain responsible for memory consolidation is stimulated by emotional arousal. Therefore, it is advised to associate learning with some emotional state like confusion or stress. Our brain's willingness to return to the equilibrium state increases its performance and cognitive abilities. Moreover, the acquired knowledge is more accessible when acquired during an emotional state. However, the emotional state must stay at mild levels; otherwise, the learning capabilities decrease. 

Another powerful tool described in this paper is a metaphor, which facilitates the process of "connecting the dots". New concepts learned metaphorically are easier to understand and recall. 

There are many more techniques increasing learning: spaced repetition, testing, flashcards, problem chunking, taking breaks, rewarding, varying environment, or switching between focus and diffuse modes. 

\bibliographystyle{alpha}
\bibliography{sample}

\end{document}
