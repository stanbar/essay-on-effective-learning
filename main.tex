\documentclass{article}

% Language setting
% Replace `english' with e.g. `spanish' to change the document language
\usepackage[english]{babel}

% Set page size and margins
% Replace `letterpaper' with`a4paper' for UK/EU standard size
\usepackage[letterpaper,top=2cm,bottom=2cm,left=3cm,right=3cm,marginparwidth=1.75cm]{geometry}

% Useful packages
\usepackage{amsmath}
\usepackage{graphicx}
\usepackage[colorlinks=true, allcolors=blue]{hyperref}

\title{Learning learning}
\author{Stanisław Barański}
\begin{document}
\maketitle

\section{Why learn how to learn}
Meritocracy is a system we live today, it grants economic goods and political power to people who brings useful skills and knowledge to society. Therefore, knowledge acquisition can be seen as an investment in our economic well-being. By increasing our learning skills we achieve better return-of-investment in both time and money spend on learning. 

\section{Mindset}
Learning can not happen without the will to learn. Without the believe that we are capable of acquiring any knowledge we will sabotage the learning process. Hence, the importance of the correct mindset, namely, growth mindset which in contrast to fixed mindset, is an believe that anything can be learned. It's just a matter of time, energy and correct techniques used to acquire it. 

Let's consider knowledge to be a baseball, a teacher to be a pitcher, a student to be a catcher and a learning process as throwing the from the pitcher to the catcher~\cite{adler2014read}. A bad pitcher may throw balls that are very hard to catch by the catcher. Therefore, bad teachers can make the process of learning knowledge harder than it actually is. However, if the student is skillful, then he will be able to catch a knowledge thrown even by a bad teacher. Hence, the vital role of the correct mindset in the process of learning. If we believe that we are responsible for the learning process as much as the teacher is. Then, we're more likely to be engaged and forgiving. Knowing that we are playing in the same team with our teacher, we may help him by interacting with him. Once we, as a students, increase our learning skills, we will be able to learn complex topics faster.


\section{Memory}
Memory is the oldest mental skill from which all others derive. Without memory we would not be able to remember new skills that we have just learned. A person who can not remember, is cut off from the knowledge of prior experiences. Without memory, logic couldn't exist, therefore we, as a society, could not develop~\cite{csikszentmihalyi1990flow}.

From the physiologically point of view, our memory is made of synapses. We create new synapses by learning new things, or strengthen existing ones by reusing them.

Hippocampus is a part of our brain that control access to our memory. This "gatekeeper" decides if the new information will be stored in our memory or not. Hippocampus is influenced by the amygdala, which in turn is influenced by emotions — phenomenea called memory enhancement effect. The more emotionally charged event, the better we remember it. If we have no emotions during learning, Amygdala is not stimulated; therefore, our hippocampus is not working correctly; therefore, we are not learning efficiently~\cite{phelps2004human, tyng2017influences}. Facts gathered without any emotional arousal are more vulnerable to disruption~\cite{doi:10.1111/1467-9280.00090}. On the other hand, "Emotion also facilitates encoding and helps retrieval of information efficiently"~\cite{tyng2017influences}. Therefore, we can boost our learning capabilities by putting ourselves in some kind of emotional state. For example, confusion is a state in which people have increased learning performance, because the bad feeling of disequilibrium of disagreeing data, arouse feeling of frustration, but also makes him seek equilibrium, which results in enhanced learning~\cite{d2014confusion}. Stress is another common emotional state associated with learning activity. Depending on intensity and duration, stress can facilitate or impair learning performance. Studies show that low level of stress can enhance cognitive capabilities, while strong stress has negative effect, especially on memory retrieval which is often observed by students underachieving at exams~\cite{vogel2016learning}.

People have two types of memory, short-term memory (working memory) and long-term memory. Working memory is responsible for gathering and encoding information. Long-term memory is responsible for storing information for later retrieval. Metaphorically speaking, working-memory is an blackboard, and long-term memory is an hard-drive. Only the information that has been persisted in long-term memory can be called learned. Therefore, in our learning, we should aim for enhancing the process of moving as much information from working-memory to long-term memory. This process of moving data from working memory to long-term memory is called consolidation. Consolidation happens when we take breaks, so sleeping, jogging, and exercising is highly advisable, not only to rest, but also to persist the accrued knowledge in long-term memory.

Learning is a process of forming new neural connections in the already existing neural network. Hence, knowledge can be seen as a map where streets represent neural connections. Learning is merging a new map with the current one we already posses. Yet, the process of merging these two maps is nothing but easy. First, the brain must apply the new map in the right location of the current map, which is a challenging task. Second, it has to correctly align the edges of these maps so that the paths are nicely connected. The fewer "hanging paths", the greater understanding because we do not have to assume where the paths go to or come from. When the paths are connected with already existing "infrastructure", we know exactly where they are coming from or going to. 


\section{Metaphore}
The concept of representing a phenomena in one domain as an instance of different phenomena in different domain is called metaphore. As we have stated previously, the process of learning has been presented as a process of throwing a baseball. By representing process of learning in such a metaphore we can apply our already existing experiences with baseball game to the new concept of learning. Therefore, saving a lot of mental resources on arranging the context of the new knowledge. If we've already seen the baseball game (and the chances are that we have), then we can reduce the time and energy needed to realize that in learning, the student is as responsible as the teacher, by prior knowledge on the catcher being as crucial as the pitcher in the baseball game.

Metaphor is a strong rhetoric device that allows knowledge extrapolation, that is, acquiring new knowledge  connecting what is new to what is known, using analogy. While learning, you can use metaphor to help your brain in connecting new knowledge to the existing one. Creating a good metaphor require creativity and energy, but the payoff is worth it. Good metaphor stays in your head for much longer than pure facts, therefore saving your energy on learning in the long term. 

\section{Conclusions}

The most fundamental in learning is the correct mindset, also called the growth mindset, which in contrast to fixed mindset, is an believe that everything can be learned. It's just a matter of time, energy and correct techniques used to acquire it. 

\bibliographystyle{alpha}
\bibliography{sample}

\end{document}
